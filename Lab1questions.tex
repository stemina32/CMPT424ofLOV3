\documentclass[12pt]{article}
 \usepackage[margin=1in]{geometry} 
\usepackage{amsmath,amsthm,amssymb,amsfonts}
 
\newcommand{\N}{\mathbb{N}}
\newcommand{\Z}{\mathbb{Z}}
 
\newenvironment{problem}[2][Problem]{\begin{trivlist}
\item[\hskip \labelsep {\bfseries #1}\hskip \labelsep {\bfseries #2.}]}{\end{trivlist}}
%If you want to title your bold things something different just make another thing exactly like this but replace "problem" with the name of the thing you want, like theorem or lemma or whatever
 
\begin{document}
 
%\renewcommand{\qedsymbol}{\filledbox}
%Good resources for looking up how to do stuff:
%Binary operators: http://www.access2science.com/latex/Binary.html
%General help: http://en.wikibooks.org/wiki/LaTeX/Mathematics
%Or just google stuff
 
\title{Homework questions for the lab 1}
\author{Celestin Masson}
\maketitle
 
\begin{problem}
{Write in LaTeX} 
Make	a	commitment	to	serious	computer	science	and	write	up	your	answers	in	LaTeX.	
Commit	both	the	.tex	source	and	the	compiled	.pdf		of	with	answers	to	GitHub.	
1. What	are	the	advantages	and	disadvantages	of	using	the	same	system	call	interface	
for	manipulating	both	files	and	devices? 
2. Would	it	be	possible	for	the	user	to	develop	a	new	command	interpreter	using	the	
system	call	interface	provide	by	the	operating	system?	How?
\end{problem}
 
\begin{proof}{1-} \newline
Since using the same system-call interface for manipulating both files and devices allows to access each device like it was a file system. It is possible to add a new device driver by simply adding the code to support the device file interface. We can conclude that one of the main advantages of using the same system-call interface for both files and devices is the development of the program code (access to device and file) and also device-driver code (API support). \newline
It can work against the user to use recurrent same interface in terms of performance and functionality. Indeed this occurs because it becomes harder to identify all the functionality of a lambda  device due to its context of the file access API. \newline
{2-} \newline 
It is possible for an user to develop a new command interpreter using the system-call interface provided by the operating system. Indeed because it is possible to create and manage processes and their communication using system calls (Similar to ability of using these functions due to an user level program with system calls).
\end{proof}

\end{document}